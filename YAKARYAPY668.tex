\chapterspecial{Yakã ryapy}{Nascentes de águas}{}
 

 

%% \versal{ILUSTRAÇÃO} 27 (Rio paraná e iguaçu)}

 

\emph{Nhamandu Tenonde} criou as seis maiores nascente de água. Estes
rios possuem riqueza abundante de peixes para os povos da Terra viverem
e extraírem seus alimentos. Desses rios, três são muito importantes para
os \emph{Mbya}, o rio Paraná, o rio Uruguai e também o rio Iguaçu, onde
ainda hoje existem, em suas proximidades, centenas de aldeias.

A Mata Atlântica é um lugar quente, onde não há geada, que fica ao redor
e à beira do mar. Por essa peculiaridade, os \emph{Guarani Mbya} deram o
nome de \emph{Yvy apy}. A Serra do Mar é chamada de \emph{Jekupe},
costas do mar, por ser a faixa litorânea de montanhas que é uma
contenção do mar e por preservar a vida neste território, sendo muito
importante espiritualmente para o povo Guarani. Em seu interior existe
uma abundância de espécies de animais silvestres e plantas medicinais
endêmicas.

%% \versal{ILUSTRAÇÃO} 28 (\versal{PÁSSARO} \versal{MONTANHA})}

 

Os rios são sagrados e têm vida em constante movimento para purificar os
seres vivos aquáticos e toda a natureza que existe em suas margens.

 
