
\chapterspecial{A tradução do Espírito}{}{Anita Ekman}
 

 

 

\epigraph{``O Guarani busca a perfeição de seu ser na perfeição do seu dizer. Nós
somos a história de nossas palavras. Tu és tua palavra, eu sou nossas
palavras. Che ko ñandeva. Potencialmente, cada Guarani é um profeta -- e
um poeta \mbox{---,} segundo o grau que alcance sua experiência religiosa.''}{\versal{Bartolomeu Melià}, \emph{Outra Palavra é Possível}} 

 

``Espirito'' e ``palavra'' são sinônimos na língua Guarani Mbya.
Nhẽe significa ao mesmo tempo ``falar'', ``vozes'', ``alma''.
Nhẽe porã, as ``boas palavras'' ou o ``espírito bom''.
Traduzir o espírito em palavras é um desafio comum ao poeta.

Porém, para um Guarani, a tradução de suas ``palavras"-almas'' para a
língua portuguesa é um desafio que transcende o literário; é em si um
ato político.

Foi o sonho de mobilizar os \emph{Guarani Mbya} para além das fronteiras
que lhes foram impostas o que me aproximou de Timóteo Verá Tupã Popygua
há mais de uma década. Juntos, visitamos inúmeras \emph{tekoas} na
América do Sul e criamos com Marcos dos Santos Tupã e Yanina Otsuka
Stasevskas um projeto em comum, batizado por Timóteo de
{Projeto} \emph{{Yvy rupa}}.

A transcrição do que foi dito na reunião de fundação desse projeto,
realizada na aldeia \emph{Tekoa Tenonde Porã}, em São Paulo, no dia 27
de Julho de 2005, diz muito sobre este livro, publicado depois de 12 anos do passo inicial dessa caminhada conjunta por \emph{Yvyrupa}.

 

\begin{quote}
``Nossa concepção indígena, principalmente Guarani, é… viver com
amplitude, espaço, não ter a fronteira, não ter divisão
geográfica… isso não existia antes… mas, quando os
\emph{Jurua} chegaram, separaram. Por exemplo, hoje, por exemplo, tem
Brasil, Argentina, Paraguai, Uruguai e Bolívia onde vivem os Guarani,
mas pro Guarani, ainda não existe fronteira… por exemplo, se você
pegar um mapa e colocá"-lo aqui, vai enxergar um linha que divide. É uma
linha imaginária. De fato, se a gente olha no chão, ela não
existe…

Antigamente, o Guarani ocupava imenso território… tinha essa
ocupação… consideram os Guarani um povo nômade…
assim… quer dizer, um povo que não para. O pessoal Guarani
vivia… assim livre… como os pássaros, como os rios…
e ele tinha território imenso em torno.

Temos que criar caminhos, sempre falo para os jovens, porque eles são o
futuro… eles sabem que têm que estar preocupados, infelizmente,
eles têm que caminhar por dois caminhos diferentes… estar na
escola, estar estudando, tendo disciplina, geografia, matemática,
história… mas, por mais que eles entendam isso, eles têm que ter
o conhecimento da essência, da sua raiz, de sua origem…

Nós temos uma história contada completamente diferente da dos
brancos… nós temos uma geografia nossa, até a formação do
universo, a formação do mundo… dos Guarani é completamente
diferente da dos homens brancos… porque existem várias
teorias… Por que tem várias teorias? Porque somos seres
humanos… somos mortais… jamais chegaremos à
conclusão… o Guarani tem força da memória, para ele extrair sua
sabedoria, seu conhecimento, ele tem que buscar isso na essência, tem
que buscar força cósmica para ele ter essência…

Sempre falo isso, cada um de nós tem mensagens dentro de nós para passar
para os próximos, cada um de nós… é micro"-cosmos,

Eu entendo assim, que cada um de nós é um mundo, somos um mundo…
um pequeno mundo… você, ele, ela, eu… todo mundo, e muitas
vezes também o que não é gente… me refiro à planta, por exemplo,
uma planta medicinal que está ali… mas, é um mundo também…
é um mundo que está ali, mas ela tem a capacidade de curar, ela tem a
capacidade de passar esta mensagem para o ser humano… Mas você
tem que saber preparar… estar ciente… nós consideramos
isso… e eu penso que jamais o Guarani perca isso… esse
conhecimento que é rico.

Sempre falo isso, que o Guarani seja Guarani sempre… daqui cem
anos, duzentos anos… por isso temos que preparar o alicerce,
preparar espaço onde as crianças vão viver… por isso, eu também
quero que a Argentina, Paraguai, Uruguai, os Guarani que vivem ali
também tenham o mesmo pensamento.

Por exemplo, na nossa espiritualidade, os pajés… eles, através da
força divina… a força da essência, através da cultura, da
religião, eles têm contato entre eles em toda a terra. Então até hoje
não tem essa fronteira, que jamais a fronteira vai dividir esse vínculo
espiritual…

Então… sempre penso isso, que é criar essa política, para o
Guarani não ter fronteiras… ir para Argentina, ter livre
acesso… ir para o Paraguai, ir para Bolívia, Uruguai,
então… não somente o contato, mas sim reconstruir a força, né,
que é a parte espiritual.

Não quero jamais que o conhecimento que temos se perca, desde Argentina,
no Paraguai, no Brasil, o rico para nós não é ter dinheiro… rico
para nós é ter sua sabedoria, conhecimento… é o dentro que é.
Assim, o território, a terra, para nós, é rico. Riqueza para nós não é
ter dinheiro, a riqueza para nós é vida, é ter saúde, é ter território,
ter seu próprio território… é viver harmonia com a natureza, pois
ela é o principio e o fim, isso é riqueza… por isso, eu quero
unir nosso povo com toda sua sabedoria na \emph{Yvyrupa}, essa terra que
é uma só e não tem fronteiras.''

\medskip 
\hfill Timóteo Verá Tupã Popygua\footnote{ Transcrição feita por Anita Ekman, Kimy Otsuka Stasevskas e Yanina
Otsuka Stasevskas.}\smallskip 

\hfill \begin{minipage}{.6\textwidth}\footnotesize
I reunião do {Projeto Yvy Rupa}, São Paulo-\versal{SP},
Brasil. 27/7/2005. 
\end{minipage}

\end{quote}

 

Durante esse percurso de 2005 a 2016, Timóteo Verá Popygua mudou de nome
para Verá Tupã Popygua, e a semente presente no pensamento
dos \emph{Mbya} participantes
do {Projeto }\emph{{Yvyrupa}} enraizou"-se no movimento
nacional da {Comissão Guarani }\emph{{Yvyrupa}}, que teve seu início entre os anos de 2002 e 2003 e foi fundado em 2006 com o apoio do {\versal{CTI} -- Centro de Trabalho Indigenista}.
Através da fundamental ajuda de Karl Werner Pothmann, Gabriela Cardozo,
Jorge Acosta e Antonio Morinigo -- o sonho
do {Projeto }\emph{{Yvyrupa}} foi realizado. Em setembro
de 2014, finalmente, reuniram"-se na aldeia \emph{Tekoa Katupyry},
em \emph{Misiones }na Argentina, os lideres \emph{Mbyá} do Brasil,
Paraguai e Argentina. Desse encontro nasceu a página
virtual guaranimbya.org, destinada a conectar as aldeias
na América do Sul e a construir, através de cartas"-vídeos, um dicionário
audiovisual Guarani. Na página é possível escutar Timóteo contando as
histórias presentes neste livro em sua língua original. 

Resultado de uma longa trajetória de amizade e companheirismo entre mim
e Timóteo, este livro expressa já em seu
título, \emph{{Yvyrupa}} {-- A Terra Uma Só}, a busca por
dissolver, passo a passo, página a página, as muitas fronteiras
(geográficas, culturais, linguísticas, políticas etc.) que separam os
demais latino"-americanos dos \emph{Guarani Mbya}. Neste livro,
testemunha"-se, sobretudo, que os \emph{Guarani Mbya}, além de serem
guerreiros na luta pela defesa de seus territórios e pela conservação da
Mata Atlântica, são também a memória viva do coração de nosso
continente.

 

 

\section{A ilustração e a realização deste livro}

\emph{Ipara} em \emph{Guarani Mbya} significa ao mesmo tempo ``desenho''
e ``escrita''.

Embora os Guarani possuam um ``sistema de notação visual'' que permite a
eles registrar informações através de grafismos milenarmente
reproduzidos em cerâmicas e cestos\footnote{ Segundo o arqueólogo Francisco Noelli, em sua dissertação de mestrado
``Sem \emph{Tekohá} não ha \emph{tekó}'', os Guarani são considerados
uma cultura prescritiva, pois mantêm há mais de 3.000 anos a mesma forma
de fazer e decorar sua cerâmica, reproduzindo padrões gráficos que lhes
conferem identidade. As ilustrações deste livro se baseiam em pesquisas
sobre tais padrões.},
a sabedoria Guarani é fundamentalmente transmitida através da palavra:
através dos cantos e pregarias proferidos na \emph{Opy}, a casa de
orações, pelos Xeramõ ´i kuery e Xejaryi Kuery, os avós sábios.

Para os \emph{Mbya}, ``a palavra falada'' significa ``oferecer o
profundo que nasce na raiz do coração. Entrega de amor que brota de um
coração e caminha para o outro''\footnote{ Memória de palavras ditas a mim pelo \emph{Mbya} \emph{Guarani} Carlos
Papa.}. É através do discurso Guarani sobre o fundamento da linguagem humana,
\emph{Ayu porã rapyta}, que podemos entender como amor e palavra nascem
de uma mesma origem divina e se expandem para formar a língua, o
pensamento e a sabedoria Guarani.

As narrativas dos \emph{Guarani Mbya} sobre a origem da terra, do ser
humano, da linguagem humana e dos animais e plantas da Mata Atlântica
foram documentadas e traduzidas pela primeira vez por León Cadogan em
\emph{Ayvú Rapyta -- textos míticos de los Mbyá"-Guaraní del
Guairá}\footnote{ \emph{Boletim 227, Antropología n\,5}, Faculdade de Filosofia Ciências e
Letras da Universidade de São Paulo, Brasil, 1959.}. Nascido em
\emph{Asunción}, Paraguai, em 1898, de pai e mãe australianos, Cadogan
dedicou"-se por mais de 40 anos à defesa dos direitos dos \emph{Guarani
Mbya}, por quem foi apelidado de \emph{Tupa Cuchuví Vevé}.

Ao longo de mais de uma década de amizade, Timóteo e eu conversamos sobre as histórias de seu povo na Yvyrupa. Foi em 2008, em uma dessas tardes regadas de mate lendo e discutindo as traduções realizadas por Cadogan do \emph{Avvy Rapta}, Timóteo revelou seu desejo de escrever um livro. Inspirada pelos poéticos comentários de Timóteo, que comparava o material compilado e traduzido por Cadogan em espanhol
com as histórias que havia escutado de seus avós e líderes espirituais,
bem como movida pela necessidade de documentar e divulgar a luta pela
Terra Guarani e a preservação da Mata Atlântica (\emph{Ka'a porã}),
ocorreu"-me a ideia de propôr a Luísa Valentini, organizadora da Coleção Mundo Indígena da Editora Hedra, que Timóteo pudesse realizar seu sonho antigo e ser o
porta-voz dos \emph{Guarani Mbya} e contar diretamente em português, sem
intermediário de um \emph{jurua} (não indígena), sua versão do
``\emph{Ayvú Rapyta''} e a história do seu povo na Yvyrupa.

Foram decisivos para a realização da publicação a revisão final de Vicente de Arruda Sampaio, bem como os recursos advindos do Programa de Ação Cultural -- Pro\versal{AC} da Secretaria da Cultura do Governo do Estado de São Paulo.

Enquanto organizadora, limitei"-me a propor a Timóteo uma divisão em
capítulos que ajudasse a nortear sua escrita. Com a primeira versão do
texto pronta, eu, ele e Yanina Otsuka Stasevskas revisamos a ortografia
e pensamos juntos como eu deveria criar as ilustrações para as
principais passagens.

Maria Inês Ladeira, antropóloga do \versal{CTI} -- Centro de Trabalho
Indigenista, ficou responsável por elaborar com Timóteo a versão final
das narrativas tradicionais. Além disso, a pesquisadora também redigiu o texto que serve de apresentação ao livro.

Fregmonto Stokes me ajudou a criar o mapa do Território Guarani. E Renan Costa Lima criou
o projeto gráfico.

Estou convencida de que, se coube a mim o desafio de organizar e,
sobretudo, ilustrar a mitologia e a história dos \emph{Guarani Mbya} --
dando rosto a seus personagens \mbox{---,} isso aconteceu graças a todos os anos
de convivência afetiva com meus amigos \emph{Guarani Mbya} nas
\emph{tekoas} da Argentina e do Brasil. É com o coração crescido de amor
e de entendimento em relação aos símbolos \emph{Mbya} que procuro passar
adiante os sinais de \emph{Nhanderu}.

Gratidão eterna a ele e a todos, \emph{Nhande'i
va'e} e \emph{Jurua,} que me deram luz e força para desenhar caminho.
Oguata Porã.

\medskip{} 

\hfill \emph{Aguyjevete.}
Jaxuka

\hfill Anita Ekman.

 

 

\section{Nota editorial}

A escrita Guarani ainda não tem uma convenção unanimemente aceita.
Adotou"-se aqui aquela que foi preferida pelo autor, Timóteo da Silva
Verá Tupã Popygua. As palavras Guarani em maiúscula referem"-se a nomes
de divindades, pessoas e entidades. Os nomes próprios compostos levam
maiúscula apenas na primeira palavra. São exceções a essa regra, em
razão da importância de toda a expressão: \emph{Nhamandu Tenondegua,
Nhamandu Tenonde}, \emph{Guarani Mbya.} Em português, também em razão da
importância das palavras, sempre há maiúsculas na grafia de ``Guarani''
(que nunca aparece no plural) e ``Terra''.
