\imagemmedia{}{./img/07.png}


\chapterspecial{Yvy tenonde}{Primeira Terra}{}
 

 

%% \versal{ILUSTRAÇÃO} 07 (5 palmeiras)}


\emph{Nhamandu Tenondegua} criou \emph{Yvy tenonde}, a primeira Terra.
Através de sua sabedoria divina, gerou \emph{te\gr{ῖ}nhirui pindovy}, cinco
palmeiras azuis. A primeira, no futuro centro da Terra, \emph{Yvy
mbyterã}. A segunda, na morada de \emph{Karai}, pai da chama. A terceira, na morada de \emph{Tupã}, pai do ar fresco e do trovão. A quarta, na origem do vento bom, \emph{Yvytu porã}, no tempo novo,
\emph{Ara pyau}. A quinta, na origem do vento frio, \emph{Yvytu ro'y},
no tempo originário, \emph{Ara ymã}. Assim, \emph{Nhamandu Tenondegua}
criou \emph{te\gr{ῖ}nhirui pindovy}, cinco palmeiras azuis{.}

Logo depois, da ponta de seu \emph{popygua}, bastão insígnia, uma
pequena porção de Terra se estendeu em cima do oceano primitivo. Então
\emph{Nhamandu Tenondegua}, pela primeira vez, desceu de seu
\emph{apyka}, assento, e pôs seus pés nessa porção redonda de Terra.
Ainda não existia a Terra em sua totalidade. Ele caminhava e olhava para
todos os lados. Não via nenhum ser vivo, mas, de repente, avistou
brotando, no centro dessa superfície, uma pequena árvore, \emph{Nhẽrumi
mirim}, que se ampliaria em floresta.
\imagemmedia{}{./img/08.png}


%% \versal{ILUSTRAÇÃO} 08 (\versal{ÁRVORE})}

Além de \emph{Nhẽrumi mirim}, viu surgir um pequeno tatu, cavando a
Terra, o primeiro entre os animais silvestres que povoariam as matas.
Viu também, voando entres os galhos, \emph{Tukanju'i}, o pequeno
passarinho primitivo, o primeiro pássaro que surgiu na extensão da
Terra.\\ Viu a pequena serpente primitiva, \emph{Mboi ymãi}, que
descansava entre as raízes de \emph{Nhẽrumi mirim}, o primeiro de todos
os répteis que existiriam no mundo. %% \versal{ILUSTRAÇÃO} 10
\imagemmedia{}{./img/09.png}

 

Assim, esses seres primitivos já anunciavam a diversidade biológica
existente no planeta.


%\imagemmedia{}{./img/11.png} 

\emph{Nhamandu Tenondegua} deu nome a esse espaço sagrado de \emph{Yvy
mbyte}, centro da Terra. Depois, com sabedoria divina, na ponta de seu
\emph{popygua}, bastão insígnia, a Terra foi se movendo \emph{mboapy
ára}, durante três dias. Seus quatros filhos, cada um com seu poder
divino, observavam e auxiliavam seu pai, \emph{Nhamandu Tenondegua}, na
criação de sua primeira morada terrestre. Quando terminou de gerar a
Terra, estendeu"-se a floresta, \emph{ka'a}. O primeiro grito de
agradecimento foi o de \emph{Nhakyrã pytãi}, cigarra vermelha.

%% \versal{ILUSTRAÇÃO} 11 (\versal{FLORESTA} cigarra vermelha)}

O primeiro animal que andou e rastejou na Terra de \emph{Nhamandu
Tenondegua} foi \emph{Mboi ymãi}, uma pequena cobra que havia surgido em
\emph{Yvy mbyte}, centro da Terra.

%\imagemmedia{}{./img/12.png}

Nas florestas não havia rios nem nascentes, então \emph{Nhamandu
Tenondegua}, com sua sabedoria divina, criou o protetor das águas,
\emph{Yamã}, girino, que fez \emph{mboapy meme} para \emph{rakã apy}, as
seis maiores nascentes e seus afluentes, que se desdobrariam em milhares
de vertentes.

%% \versal{ILUSTRAÇÃO} 12 (\versal{FLORESTA} \versal{GIRINO} \versal{NASCENTES})}


Depois que criou \emph{Yamã}, o protetor das nascentes, \emph{Nhamandu
Tenondegua} viu que havia florestas, mas não planície ou campos
naturais. Então criou \emph{Tuku ovy}, gafanhoto azul, para ele pôr
seus ovos no chão e fazer brotar os grandes campos e planícies. O primeiro que cantou em agradecimento, no meio da planície, foi
\emph{Ynambu pytã}, nambu vermelho.

%% \versal{ILUSTRAÇÃO} 13 (nambu)}

%\imagemmedia{}{./img/13.png}

 

Quem fez o primeiro buraco, uma toca para criar filhotes na Terra de
\emph{Nhamandu Tenondegua}, foi \emph{x\gr{ῖ}guyre}, tatu, e seus filhotes se
espalharam por vários lugares do mundo.

Depois de ter criado tudo, \emph{Nhamandu} terminou seu trabalho,
\emph{Yvyrupa}, a Terra. Com sua sabedoria divina, subiu até \emph{Oyva
ropy}, seu firmamento, para poder descansar e cuidar de suas
criações.
