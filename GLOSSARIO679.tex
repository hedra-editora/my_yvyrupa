\chapterspecial{Glossário}{}{}
 

\begin{itemize}
\itemsep1pt\parskip0pt\parsep0pt
\item
  aguyje: \emph{nossa plenitude} de viver
\item
  Aguyjevete: \emph{saudações}
\item
  Apyka: \emph{assento}
\item
  Ara pyau: \emph{tempo novo, tempo da renovação, primavera, tempo das
  chuvas e do calor, tempo de plantio; o} \emph{tempo em que Nhanderu
  Nhamandu se levantou}
\item
  Ara ymã: \emph{tempo originário, que sempre ressurge e anuncia a
  chegada do frio e do recolhimento;} \emph{tempo primordial de descanso
  de Nhanderu Nhamandu Tenondé em seu desdobramento, tempo"-espaço, de
  março a setembro} 
\item
  avaxi ete'i: \emph{milho verdadeiro}
\item
  ayu porã rapyta: \emph{origem das belas palavras}
\item
  ayu porã: \emph{as belas palavras,} \emph{palavras que animam}
\item
  eterã: [\emph{complemento a outra palavra que indica o futuro}]
\item
  Guarani Mbya: \emph{coração bom}, Guarani
\item
  guavira mir\gr{ῖ}: \emph{gabiroba do campo}
\item
  guembe: \emph{araçá mirim, pequena goiaba do brejo}
\item
  guyra: \emph{as aves}
\item
  Jakaira ru eterã: \emph{pai da neblina}
\item
  Jakaira xy eterã: \emph{mãe divina dos filhos da neblina}
\item
  jaxukava porãgue'i: \emph{os homens}, Mbya
\item
  jeguaka poty yxapy rexa: \emph{cocar, flor, orvalho, olho; o cocar de
  plumagens enfeitado com orvalho de flores, surgido no meio da noite
  originária}
\item
  jeguakava \emph{e} jaxukava porãgue'i: \emph{filhos e filhas dos}
  \emph{nhẽe}, \emph{homens e mulheres, a humanidade}
\item
  jerojy: \emph{dança ritual}
\item
  jety mir\gr{ῖ}i: \emph{batata doce original}
\item
  joguero guata porã: \emph{caminhada} \emph{sagrada dos Guarani
  pelos} \emph{caminhos revelados} 
\item
  jurua: \emph{não"-indígenas; literalmente} \emph{``boca} \emph{com
  cabelo''} \emph{(}juru \emph{= boca;} a \emph{=} \emph{cabelo), em
  referência} \emph{à} \emph{barba} \emph{e} \emph{ao} \emph{bigode} \emph{dos} \emph{europeus
  à época da conquista} 
\item
  ka'a mir\gr{ῖ}\emph{: erva mate}
\item
  ka'a: \emph{floresta}, mata
\item
  ka'aguy karape: \emph{matas baixas}
\item
  ka'aguy yvyty regua: \emph{toda a região da Serra do Mar} 
\item
  ka'arua: \emph{onde o sol se põe; uma das 5 direções da Terra},
  \emph{Oeste} 
\item
  Karai ru eterã: \emph{pai das chamas}
\item
  Karai xy eterã: \emph{mãe divina dos filhos da chama}
\item
  Kuaray: \emph{o sol}
\item
  kuri: \emph{araucária}
\item
  kurity: \emph{pinheirais de araucária}
\item
  maino ou mainomby: \emph{colibri, beija"-flor, o pássaro originário}
\item
  mandi'o mir\gr{ῖ}i: \emph{espécie de mandioca}
\item
  manduvi mir\gr{ῖ}i: \emph{amendoim original}
\item
  mandyju mir\gr{ῖ}i: \emph{algodão original}
\item
  marã e'yrã: \emph{eternidade, tudo o que não pode ser destruído}
\item
  mba'emo ka'aguy regua: \emph{os animais silvestres}
\item
  mbaraete: \emph{força física e mental}
\item
  mboapy ára: \emph{durante três dias}
\item
  Mboi ymãi: \emph{pequena cobra, primeiro animal que andou e rastejou
  sobre a Terra; espécie de serpente da mata atlântica conhecida
  popularmente como ``cobra rainha''}
\item
  mborai: \emph{canto sagrado}
\item
  mborayu: \emph{o amor infinito}
\item
  Guarani Mbya: \emph{povo Guarani}
\item
  mymba retã, ypo, guyra: \emph{morada de animais e aves marinhas}
\item
  Nhakyrã pytãi: \emph{cigarra vermelha}
\item
  Nhamandu ombojera: \emph{nosso pai}
\item
  Nhamandu py'aguaxu: \emph{Nhamandu que possui coragem, força de
  elevação espiritual}
\item
  Nhamandu py'aguaxu xy eterã: \emph{mãe divina de} Kuaray, \emph{o sol}
\item
  Nhamandu py'aguaxu: \emph{o sol} (= Kuaray)
\item
  Nhamandu ruete tenondegua: \emph{pai primeiro do sol divino}.
\item
  Nhamandu Tenondegua: \emph{primeiro pai divino, criador
  originário} \emph{dos seis firmamentos e de} Yvy,
  \emph{Terra}; \emph{pai de} Kuaray,{} Jakaira, Karai, Tupã
\item
  Nhande'i va'e: \emph{antigo nome do} Guarani Mbya
\item
  Nhandereko: \emph{nosso modo de ser}
\item
  Nhanderu tenondegua: \emph{nosso pai primeiro}
\item
  Nhanderu: \emph{pai} \emph{divino}
\item
  Nhẽe ru ete: \emph{espíritos originais} (Nhamandu Tenondegua, Kuaray,
  Karai, Jakaira, Tupã)
\item
  nhẽe: \emph{espíritos,} \emph{palavras, sons (de pássaros)}
\item
  Nhẽrumi mirim: \emph{pequena árvore da qual provêm todas as
  florestas,} \emph{espécie de árvore da Mata Atlântica conhecido
  popularmente como ``vassourinha''}
\item
  nhuũ upa: \emph{campos}
\item
  onhembojera: \emph{desdobramento, desabrochando, surgirmento}
\item
  opy: \emph{casa de rituais}
\item
  opyre: na \emph{casa de rituais}
\item
  ore retarã ypykuery: \emph{nossos antigos parentes; os parentes
  originários}
\item
  Oyva ropy: \emph{firmamento de} Nhamandu Tenondegua, \emph{para onde
  ele se retira após criar a Terra para descansar e cuidar de suas
  criações}
\item
  Oguata Porã: \emph{caminhos revelados}
\item
  pakuri e ka'aguy poty: \emph{árvore frutífera} e \emph{plantas
  medicinais}
\item
  Para guaxu rembe: \emph{beira do Oceano Atlântico (}para =
  \emph{oceano}; guaxu = \emph{grande}; rembe = \emph{margem), lugar em}
  \emph{Tenondere}, \emph{onde nasce o sol, também chamado de} Yvy porã
\item
  Para guaxu: \emph{o grande mar, o Oceano Atlântico}
\item
  Para ja'o rakã: \emph{ilhas}
\item
  Para pyxi rakã: \emph{mangue}
\item
  Pará rakã apy: \emph{as nascentes}
\item
  Para rembe: \emph{à margem do mar}
\item
  Para yvytu ro'y: Para = \emph{oceano}, yvytu = \emph{vento}, ro'y =
  \emph{frio}
\item
  Parana: \emph{rio Paraná}
\item
  Paranapuã: \emph{ondas do mar}
\item
  Pararakã: \emph{ramos dos grandes rios, afluentes}
\item
  Paraygua: \emph{rio} \emph{Paraguai}
\item
  pekuru: \emph{tipo de bambu}
\item
  petỹ: \emph{fumo}
\item
  petỹgua: \emph{cachimbo que} \emph{recebe} \emph{as emanações de} ayu
  porã (\emph{belas palavras})
\item
  pindovy: \emph{palmeira azul}
\item
  popygua: \emph{bastão insígnia de} Nhamandu
  tenondegua\emph{,} \emph{objeto sagrado utilizados pelos} nhande'i vae
\item
  porã: \emph{bonito, bom}
\item
  porãgue'i: \emph{nossos filhos queridos}
\item
  tajy poty: \emph{ipê amarelo, que, quando floresce, anuncia a chegada
  de Ara Pyau}
\item
  Takua ru ete: \emph{pai da purificação dos espíritos, dos} nhẽe
\item
  Takua ete: \emph{o espaço sagrado que o espirito} Guarani \emph{se
  purifica, e para onde vai o espirito quando o corpo morre}
\item
  takua, takuara: \emph{taquara}
\item
  takuarembo: \emph{criciúma} \emph{(espécie nativa da Mata Atlântica de
  bambu)}
\item
  takuaruxu: \emph{tipo de bambu liso}
\item
  tapyi = tekoa: \emph{lugares onde acontece nosso próprio modo de vida}
\item
  tataryku: \emph{fogo líquido, lavas de vulcão}
\item
  Tataypy rupa: \emph{as aldeias, onde se acende o fogo; lugares
  renascentes em} Yvyrupa; a \emph{ocupação} Guarani
\item
  tekoa = tapyi: \emph{lugares onde acontece nosso próprio modo de vida}
\item
  Tenonde: \emph{frente}, \emph{onde o sol nasce; uma das 5 direções da
  Terra}
\item
  Tenondere: \emph{para frente}
\item
  te\gr{ῖ}nhirui: \emph{Cinco, dois pares e um impar}
\item
  te\gr{ῖ}nhirui pindovy: \emph{cinco palmeiras azuis}
\item
  Tukanju'i: \emph{pequeno passarinho primitivo, o primeiro pássaro que
  surgiu na da Terra}
\item
  Tuku ovy: \emph{gafanhoto azul, que põe ovos no chão e faz brotar os
  grandes campos e planícies}
\item
  Tupã ru eterã: \emph{pai do trovão, do vento e da brisa}
\item
  Tupã xy eterã: \emph{mãe divina dos filhos do trovão, das chuvas e do
  vento}
\item
  Uruguay: \emph{rio Uruguai}
\item
  urukure'a: \emph{coruja}
\item
  xejaryi (kuery): \emph{nossas avós, anciãs}
\item
  xeramõi (kuery): \emph{nossos avôs, anciãos}
\item
  xeramõi: \emph{anciãos, velhos com sabedoria}
\item
  ximbo: \emph{tipo de cipó}
\item
  x\gr{ῖ}guyre: \emph{tatu, animal que fez o primeiro buraco, uma toca para
  criar filhotes na Terra}
\item
  ya para'i: \emph{melancia}
\item
  Yakã ryapy: \emph{olho d'água}
\item
  Yakã: \emph{rios}
\item
  Yamã: \emph{girino, o protetor das nascentes}
\item
  Yguaxu: \emph{rio Iguaçu}
\item
  Ynambu pytã: \emph{nambu vermelho}
\item
  Yro'y rapyta: \emph{a origem do frio, o Oceano Pacífico}
\item
  Yta jekupe: \emph{contenção do mar}
\item
  Ytajekupe: \emph{Cordilheira dos Andes, muralha de pedras}
\item
  ytaku'i: \emph{areia}
\item
  yvapuru: \emph{jabuticaba}
\item
  Yvy apy: \emph{ponta da Terra, Mata Atlântica ou Serra da do Mar;
  chamada de} Jekupe\emph{, costas do mar, por ser a faixa litorânea de
  montanhas; lugar muito importante espiritualmente para o povo
  Guarani.} 
\item
  Yvy mbyte: \emph{centro da Terra, região hoje conhecida como Tríplice
  Fronteira; uma das 5 direções da Terra}
\item
  Yvy mbyterã: \emph{centro da Terra}
\item
  Yvy porã: \emph{Terra boa, fértil e aconchegante}
\item
  Yvy tenonde: \emph{primeira Terra}
\item
  Yvy: \emph{Terra.} \emph{A palavra} Yvy \emph{é formada por} yy\emph{,
  que indica que a terra se forma através da água}
\item
  Yvyjuky: \emph{Terra feita de sal}
\item
  yvyku'ix\gr{ῖ}renda: \emph{morada de Terra branca}
\item
  yvyrá: \emph{árvores e plantas,} \emph{o que nasce da terra} yvy
\item
  Yvyrupa: \emph{usado para definir o território tradicionalmente
  ocupado pelos} Guarani Mbyá \emph{e também, em um sentido mais amplo,
  o próprio planeta Terra}
\item
  Yvytu katu: \emph{onde se originam os ventos bons; uma das 5 direções
  da Terra}
\item
  Yvytu porã: \emph{sopro da terra, ventos, origem dos ventos bons,
  ventos originários}
\item
  Yvytu ro'y: \emph{origem do vento frio}
\item
  Yvytu ymã: \emph{lugar dos ventos originários, frios; uma das 5
  direções da Terra}
\item
  Yvytu yma'\gr{ῖ}: \emph{vento originário}
\item
  Yvyty: \emph{muitas montanhas}
\item
  Yy\emph{: água,} \emph{líquido, corrente}
\item
  Yy ramõi: \emph{onde nasce o sol, chamado também de} Para guaxu\emph{,
  o grande mar, o Oceano Atlântico}
\item
  Yyguaa py: \emph{encontro de rio com o mar, atual Iguape}
\item
  Yyrupamarãe'y: \emph{lagos de águas eternas}
\item
  Yyupa: \emph{grande poça de água, atual Lagoa dos Patos}
\end{itemize}

\emph{​}\versal{OBS}: Este glossário foi formulado pelos editores e revisado por
Timóteo da Silva Verá Popygua.
