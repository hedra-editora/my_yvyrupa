\imagemmedia{}{./img/2.png}

\chapterspecial{Nhanderu nhamandu tenondegua}{Nhamandu, nosso primeiro pai}{}
 

\begin{verse}
Uma luz infinita\\
surge.\\
Através da noite originária,\\
surge\\
\emph{Nhamandu Tenondegua},\\
nosso primeiro pai divino,\\
com sabedoria infinita\\
e com amor infinito.\\!
 

\emph{Nhamandu} gerou \emph{apyka},\\
assento divino.\\
Nele surge o cocar divino de plumas,\\
enfeitado com orvalho de flores:\\
\emph{jeguaka poty yxapy rexa.}\\!

Por entre as plumagens de flores,\\
\emph{maino},\\
o pássaro primitivo,\\
o colibri,\\
voa no meio da noite originária.\\!
\end{verse}

\imagemmedia{}{./img/03.png}

Uma luz infinita vem da sabedoria divina e do amor infinito
de \emph{Nhamandu.}

 

\asterisc{}

%% \versal{ILUSTRAÇÃO} 03 (\versal{PÁGINA} \versal{ILUSTRAÇÃO} \versal{COLIBRI})} 

Enquanto \emph{Nhamandu}, o primeiro pai divino, \emph{onhembojera}, se
desdobrava na noite originária, ele ainda não sabia como seria o futuro
do universo e do firmamento e onde seria sua futura morada celeste.
Enquanto isso, o colibri, o pássaro primitivo oferecia o néctar do
orvalho das flores para alimentar o seu criador \emph{Nhamandu}.

Nosso pai \emph{Nhamandu} ainda não havia gerado a Terra. Mesmo não
havendo sol, \emph{Nhamandu}, o detentor da aurora, iluminava a noite
originária com a luz do seu próprio coração. Com sua sabedoria divina, o
verdadeiro pai \emph{Nhamandu} vivia no meio do vento originário e
descansava. Enquanto isso, fazendo a escuridão, \emph{urukure'a}, a
coruja, dá origem ao crepúsculo e à noite.

%% \versal{ILUSTRAÇÃO} 04 (\versal{PÁGINA} \versal{ILUSTRAÇÃO} \versal{CORUJA})} 
 

Nosso pai verdadeiro, \emph{Nhamandu}, ainda não havia criado sua morada
celeste e também ainda não havia criado a primeira Terra, \emph{Yvy
tenonde}. Vivia no meio do vento originário, \emph{Yvytu yma'\gr{ῖ}.} Nesse
lugar, nosso pai ficou durante o seu desdobramento.

Da sabedoria de \emph{Nhamandu}, da sua chama e da sua neblina divina,
nascem as belas palavras, \emph{ayu rapyta}. Ele é o dono da palavra.
Ainda não existe a Terra, nem mesmo todas as coisas que vão se
reproduzir no mundo. Todavia, permanece a noite primitiva.

Depois de ter criado a origem das belas palavras, \emph{Nhamandu} criou
a fonte do amor infinito e \emph{mborai}, o canto sagrado. A Terra ainda
não existe, permanece a noite primitiva.

Nhamandu, depois de ter criado as três origens divinas -- \emph{ayu porã
rapyta}, a origem das belas palavras, \emph{mborai}, o canto divino,
\emph{mborayu miri}, o amor infinito , gerou aqueles com quem iria
dividir estas três fontes divinas de sabedoria infinita.
