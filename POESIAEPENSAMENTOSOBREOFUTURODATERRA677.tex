%\imagemmedia{}{../img/31.png}

\chapterspecial{Poesia e pensamento sobre o futuro da terra}{}{}
 

%% \versal{ILUSTRAÇÃO} 31 \versal{MAPA}}

 

O nosso planeta é um grande jardim de \emph{Nhanderu}. Devemos cuidar
dele, não o destruir, para que nosso futuro possa ser maravilhoso, sem
preconceitos, sem covardia, somente amor e fraternidade. \emph{Nhanderu}
criou o grande \emph{tekoa} onde acontece nosso modo de vida humana.

Vidas têm essência, palavras têm donos, e devemos ser solidários uns com
os outros. Assim podemos viver plenamente no jardim de \emph{Nhanderu},
pois somos simplesmente transitórios. Precisamos deixar esse legado aos
nossos filhos e netos, para que seja o mundo cheio de paz e harmonia
entres todos os povos.

Os \emph{Guarani Mbya} descobriram este lugar há milhares de anos atrás.
Todo este território pertence ao povo \emph{Guarani Mbya}. Nossa
cosmovisão reafirma esse fato. Portanto, queremos que nosso direito de
ser e de viver nesta Terra, de acordo com nossos costumes, princípios e
tradições seja respeitado pela sociedade não indígena.

\medskip{} 

\hfill\versal{AGUYJEVETE}
\hfill Timóteo da Silva Verá Tupã Popygua
