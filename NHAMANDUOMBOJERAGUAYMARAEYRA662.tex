\chapterspecial{Nhamandu ombojera gua'y marã e'yrã}{«Nhamandu» concebe seus filhos para a eternidade}{}
 

 

Através de seu poder e sabedoria divina, Nhamandu concebeu seu filho de
coração grande, \emph{Nhamandu py'aguaxu}, \emph{Kuaray}, o Sol, com seu
grande poder de iluminação, para ser o pai dos espíritos, dos
\emph{nhẽe}, de seus filhos e filhas, \emph{jeguakava} e \emph{jaxukava}
\emph{porãgue'i}, que irão nascer na Terra de \emph{Nhamandu}.

\emph{Nhamandu Tenondegua}, com sua sabedoria divina, gerou também os
seus filhos guardiões das fontes divinas:

\emph{Jakaira ru eterã}, o futuro pai da neblina

\emph{Karai ru eterã}, o futuro pai das chamas,

\emph{Tupã ru eterã}, o futuro pai do trovão, do vento e da brisa.

Em seguida, \emph{Nhamandu Tenonde}, com o amor infinito e sabedoria
divina, criou \emph{Nhamandu py'aguaxu xy eterã}, aquela que seria a mãe
divina de \emph{Kuaray}, o Sol, concedendo"-lhe o mesmo poder divino e a
sabedoria infinita.

\emph{Jakaira ru eterã}, pai da neblina, com o amor infinito e a
sabedoria divina que recebeu de seu pai \emph{Nhamandu Tenondegua},
criou \emph{Jakaira xy eterã}, a futura mãe divina dos filhos da
neblina, concedendo"-lhe o mesmo poder divino e a sabedoria infinita

\emph{Karai ru eterã}, pai da chama, com o amor infinito e a sabedoria
divina que recebeu de seu pai \emph{Nhamandu Tenondegua}, criou
\emph{Karai xy eterã}, a futura mãe divina dos filhos da chama,
concedendo"-lhe o mesmo poder divino e sabedoria infinita.

\emph{Tupã ru eterã}, pai do trovão, das chuvas e do vento, com o amor
infinito e a sabedoria divina que recebeu de seu pai \emph{Nhamandu
Tenondegua}, criou \emph{Tupã xy eterã}, a futura mãe divina dos filhos
do trovão, das chuvas e do vento, concedendo"-lhe o amor e a sabedoria
infinita.

\emph{Nhamandu Tenondegua}, depois de dividir a sabedoria das origens do
amor, \emph{mborayu}, do canto sagrado, \emph{mborai}, e das belas
palavras, \emph{ayu porã}, consagra os seus filhos como guardiões das
fontes divinas. Ainda não existia a Terra, permanece a noite primitiva.

Antes de criar \emph{Yvy}, Terra, \emph{Nhamandu} criou a morada celeste
com seis firmamentos. Quatro, pertencem aos seus filhos --
\emph{Kuaray}, \emph{Jakaira}, \emph{Karai}, \emph{Tupã}. Apesar de
possuírem poder divino, eles nada farão sem o aconselhamento de seu pai
\emph{Nhamandu Tenondegua}. Restaram dois firmamentos. Um deles ficou
com \emph{Takua ru ete}, pai da purificação dos espíritos, dos
\emph{nhẽe}. O outro pertence à \emph{Nhamandu ruete tenondegua}. É onde
fica o assento divino, \emph{apyka}, com \emph{jeguaka poty yxapy rexa},
um cocar de plumagens enfeitado com orvalho de flores, que surgiu no
meio da noite originária. \emph{Mainomby}, colibri, o pássaro
originário, também está no último firmamento com seu criador,
\emph{Nhamandu Tenondegua.} 

A língua dos \emph{jeguakava} e \emph{jaxukava} \emph{porãgue'i} nasce
de \emph{ayu} \emph{porã rapyta}, a origem das belas palavras.

%% \versal{ILUSTRAÇÃO} 06 (\versal{FILHO} \versal{DA} \versal{CHAMA})} 

 
