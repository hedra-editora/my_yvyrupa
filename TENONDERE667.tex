\imagemmedia{}{./img/21.png}

\chapterspecial{Tenondere}{À nossa frente, onde nasce o sol}{}
 

%% \versal{ILUSTRAÇÃO} 21 (\versal{ONDE} \versal{NASCE} O \versal{SOL})}

 

\emph{Xeramõi} convoca a todos para continuarem a caminhada para
alcançar \emph{Tenondere}, onde nasce o sol, em \emph{Yy ramõi}, chamado
também de \emph{Para guaxu}, o grande mar, o Oceano Atlântico. Para
realizarem essa caminhada, \emph{ore retarã ypykuery}, nossos parentes
originários, levavam com eles suas variedades de plantas originais, que
foram colocadas por \emph{Nhanderu Tenondegua} em \emph{Yvy mbyte}:
\emph{jety mir\gr{ῖ}i}, batata doce original, \emph{avaxi ete'i}, milho
verdadeiro, \emph{manduvi mir\gr{ῖ}i}, amendoim original, \emph{mandyju
mir\gr{ῖ}i}, algodão original, \emph{mandi'o mir\gr{ῖ}i}, espécie de mandioca,
\emph{ya para'i}, melancia, \emph{petỹ}, fumo, \emph{ka'a}, erva mate, e
muitas outras plantas. Levavam em forma de alimentos e de sementes.

Para chegar à margem do Mar, \emph{Yvy apy}, ponta da Terra, andaram
primeiramente na direção de \emph{Yvytu ymã}, lugar dos ventos
originários. Passaram por vários \emph{nhuũ upa}, campos, \emph{kurity},
pinheirais de araucária, \emph{ka'aguy karape}, matas baixas,
encontraram \emph{guavira mir\gr{ῖ}}, gabiroba do campo, e muitas plantas que
já conheciam. \emph{Nhanderu} indicava os lugares onde deveriam parar e
cultivar as sementes e os frutos trazidos para se reproduzirem em todos
os cantos de \emph{Yvyrupa}, a Terra criada por ele.

Durante a caminhada \emph{Nhanderu} revelava onde poderiam encontrar
\emph{takua}, taquara, \emph{pekuru}, tipo de bambu, \emph{takuaruxu},
tipo de bambu liso, \emph{takuarembo}, e outras espécies de
\emph{takuara}. Revelava também onde encontrar \emph{guembe}, araçá
mirim, pequena goiaba do brejo, \emph{pakuri} e \emph{ka'aguy poty},
plantas medicinais, em toda a extensão da Terra criada por
\emph{Nhamandu} \emph{Tenondegua}.

%% \versal{ILUSTRAÇÃO} 22 (\versal{PLANTAS} E \versal{ALDEIAS})}

 

Seguiam às margens de vários \emph{yakã}, rios, que nomearam
\emph{Yguaxu}, Iguaçu, \emph{Parana}, Paraná, \emph{Paraygua}, Paraguai
e \emph{Uruguay}, Uruguai, que deságua no mar, e também deram nomes a
muitos \emph{pararakã}, ramos dos grandes rios, afluentes.

E encontraram \emph{Yyupa}, grande poça de água, Lagoa dos Patos. Em
cada lugar que chegavam, através da sabedoria espiritual, verificavam
que existiam \emph{yvyra}, árvores e plantas semelhantes ou idênticas,
como também eram idênticos \emph{mba'emo ka'aguy} \emph{regua}, os
animais silvestres,{} e \emph{guyra}, as aves. Descobriram as
várias plantas medicinais, árvores frutíferas e várias espécies de
animais deixados por \emph{Nhamandu Tenonde}.

Seguindo o rumo de \emph{Kuaray}, Sol, chegaram em \emph{Yta jekupe},
contenção do mar, \emph{ka'aguy yvyty regua}, Serra do Mar, um lugar
quente e muito exuberante, com muitos animais. Até que finalmente chegam
a \emph{Para rembe}, à margem do Mar, \emph{Tenondere}, lugar onde nasce
o sol.

%% \versal{ILUSTRAÇÃO} 23 (\versal{RIO} \versal{DE} \versal{JANEIRO})}

 

Para \emph{ore retarã ypykuery}, nossos parentes originários, chegar à
beira do Oceano Atlântico era a grande esperança, porque \emph{Nhanderu}
havia revelado que ali era \emph{Yvy porã}, Terra boa e aconchegante.
Este lugar, em \emph{Tenondere}, onde o sol nasce, chamamos de
\emph{Para guaxu rembe}, porque \emph{Para} é ``oceano'', \emph{guaxu} é
``grande'' e \emph{rembe} é ``margem''. Assim como \emph{Yvy mbyte}, o
centro da terra, \emph{Ka'arua}, o lugar onde o sol se põe, \emph{Yvytu
ymã}, o lugar dos ventos originários, frios, e \emph{Yvytu} \emph{katu},
o lugar onde se originam os ventos bons, \emph{Para guaxu rembe} também
é de muita inspiração para nos fortalecermos espiritualmente, para
formar \emph{tekoa}, onde acontece nosso modo de vida, para viver o
\emph{nhandereko}, nosso modo de ser, para ter \emph{yvy poty aguyje},
agricultura e plantio com abundância, e para \emph{oupyty aguã Nhanderu
arandu}, para alcançar a sabedoria divina, a morada dos \emph{Nhanderu}.

\emph{Ore retarã ypykuery}, nossos parentes originários, através da
sabedoria espiritual e da revelação de \emph{Nhanderu}, andaram pela
beirada do Mar. Onde ficavam, formavam \emph{tekoa} e davam nomes aos
lugares, como: \emph{Para ja'o rakã}, ilhas, \emph{Yyguaa py}, encontro
de rio com o mar, Iguape, \emph{Para pyxi rakã}, mangue,
\emph{Paranapuã}, ondas do Mar.

%% \versal{ILUSTRAÇÃO} 24 (\versal{ILHAS})}

 

Os tempos se passaram e \emph{jeguakava} e \emph{jaxukava porãgue'i}
continuaram a realizar suas longas caminhadas, levando, trazendo e
plantando diversas sementes e criações de \emph{Nhanderu} para povoar
\emph{Yvy mbyte} e \emph{Para guaxu rembe}: \emph{yvapuru}, jabuticaba,
\emph{kuri}, araucária, \emph{Ka'a mir\gr{ῖ}}, erva mate, e também urtigas,
medicinais, diversas palmeiras, \emph{ximbo}, tipo de cipó, entre outras
plantas usadas como alimento e remédio.

E assim adquiriam e transmitiram aos seus descendentes uma vasta
sabedoria milenar sobre as florestas que formam a Mata Atlântica para
enriquecer \emph{Yvyrupa}, nosso território tradicional.



%% \versal{ILUSTRAÇÃO} 25 (Mata Atlântica)}
\imagemmedia{}{./img/25.png}

 

\part[Lugares renascentes em Yvyrupa]{Tataypy Rupa — Lugares renascentes em Yvyrupa (a ocupação Guarani)}
\chapter*{}  

Nós, \emph{Nhande'i va'e}, conhecidos atualmente como \emph{Guarani}
\emph{Mbya}, sabemos, há milhares de anos, que a Terra em que vivemos é
redonda. Nessa Terra, famílias inteiras, mulheres, homens e crianças,
seguindo a orientação de \emph{Nhanderu tenonde}, continuam caminhando,
às vezes durante meses, anos e até mesmo décadas, para povoar, ocupar,
cuidar e renovar \emph{Yvyrupa}.

Nós, \emph{Mbya}, desde o surgimento, sempre ocupamos as regiões de Mata
Atlântica, formando vários \emph{tapyi}, ou \emph{tekoa}, lugares onde
acontece nosso próprio modo de vida.

 
