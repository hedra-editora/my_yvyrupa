\chapter{Introdução}

Pelas linhas e entrelinhas deste livro, recebemos as palavras milenares,
consagradas e cultivadas oralmente pelos \emph{Nhande’i va’e, jeguakava e jaxukava,}
contemporaneamente conhecidos como \emph{Guarani Mbya}.

No acervo de fontes
inesgotáveis de sabedorias que brotam sucessiva e simultaneamente em várias
partes de \emph{Yvyrupa} e que jamais serão totalmente capturáveis, cosmologia,
ciência, ética e percepção se interagem e fornecem os fundamentos para a
expressão verbal do pensamento guarani.

A meu ver, o ineditismo deste lindo
trabalho está na dedicação do autor, Verá Tupã Popygua, também conhecido como
Timóteo – nome adotado para ser identificado no mundo não indígena –, à
elaboração de uma versão escrita de parte da sofisticada literatura oral do
povo Guarani.

Pesquisador atento, ele não se satisfez em transcrever as falas
inspiradas dos profetas ou em reproduzir aquelas traduzidas em obras literárias
e etnográficas de autores como León Cadogan, nas quais identificou as palavras
escutadas em \emph{Yvy mbyte}, no centro da terra, e em \emph{Yvy apy}, na ponta da terra,
nas aldeias situadas na costa atlântica. Verá traz para a linguagem escrita seu
próprio diálogo com a bibliografia existente e as belas palavras que escutou
dos \emph{xeramõ’i}, ao longo da vida, convertendo-o num texto original.

Pode-se
dizer que, em sua saga de escritor, Timóteo Verá trilhou o caminho das pedras.
Superou dificuldades e obstáculos, e encorajou-se para atingir um público de
leitores que contempla também os leigos acerca da cosmologia e da cosmografia
guarani. Com profundidade e leveza, transpôs os muros do indecifrável das
palavras proferidas pelos sábios, transformando-as em narrativa escrita. O
acerto no uso do vocabulário em português, que possibilitou elucidar concepções
e expressões muitas vezes intraduzíveis, revela a pesquisa apurada do autor em
busca de recursos semânticos que propiciem aos não indígenas (jurua, etava’e
kuery)\footnote{\emph{Jurua} significa literalmente “boca com cabelo” (juru = boca; a
    = cabelo), em referência à barba e ao bigode dos europeus à época da
    conquista.  \emph{Etava’e kuery} significa “aqueles que são muitos no mundo”.} o
acesso aos princípios da origem do mundo e da sociedade Guarani Mbya.  Neste
livro, entrevemos também a intenção de Verá Popygua em mostrar como e por que
os Guarani continuam cuidando de \emph{Yvyrupa}, a plataforma terrestre sobre a qual
Nhamandu edificou o mundo para ser povoado, percorrido e compartilhado pelos
Guarani. E é dessa forma que, em suas próprias palavras, Verá exprime o sentido
do pertencimento à terra que lhes pertence. Terra concebida e percebida a
partir de um tempo primordial que se eterniza, pois se atualiza com a renovação
dos ciclos da vida. Brinda-nos assim, ao revelar como vivem sua cosmologia
aqueles que continuam a ser \emph{Nhande’i va’e}, os filhos de Nhamandu.  No suporte
da palavra escrita, construindo pontes entre significados, o autor revela sua
expertise em desvendar o mundo a partir da cosmologia guarani. Mundo ao qual
também pertencemos por compartilhar, embora de forma desigual, um mesmo
território.

Quem dera, chegue o tempo em que \emph{etava’e kuery} possam ser dignos
de reverter o quadro de extermínios e destruições promovidos durante séculos e
passar a figurar como aliados na conservação da terra em que pisamos.


\medskip\hfill\emph{Ha’evete!}

\bigskip
\hfill Maria Inês Ladeira
