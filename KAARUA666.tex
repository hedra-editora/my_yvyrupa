


\part[Desde o início do mundo]{Oguata porã — Desde o início do mundo, joguero guata porã, eles percorreram os caminhos revelados}

\imagemmedia{}{./img/18.png}

\chapterspecial{Ka'arua}{Onde o sol se põe}{}
%% \versal{ILUSTRAÇÃO} 18 (\versal{SOL} \versal{SE} PÕE)}

Através da iluminação dos \emph{Nhẽe ru ete} -- \emph{Nhamandu
Tenondegua}, \emph{Kuaray}, \emph{Karai}, \emph{Jakaira}, \emph{Tupã}
\mbox{---,} \emph{ore retarã ypykuery}, nossos antigos parentes, iniciaram a
caminhada já sabendo o que iriam encontrar pela frente. Saíram de
\emph{Yvy mbyte}, centro da Terra, e caminharam em direção ao sol poente
para chegar à margem do mar.

Eles descobriram pela frente muitas montanhas, \emph{yvyty}, de difícil
acesso, lugares íngremes aos quais deram o nome \emph{Ytajekupe},
Cordilheira dos Andes, muralha de pedras.



%% \versal{ILUSTRAÇÃO} 19 (\versal{CORDILHEIRA} \versal{DOS} \versal{ANDES})}

 

Os tempos passaram, e eles chegaram a outro lugar, e descobriram que ali
tinha sal e chamaram de \emph{Yvyjuky}, Terra feita de sal. \emph{Ore
retarã ypykuery}, nossos antigos parentes, não permaneceram ali porque
não era bom para o plantio, nem mesmo bom para formar o \emph{tekoa}, e
continuaram a caminhar. Em cada lugar que chegavam davam um nome.

\imagemmedia{}{./img/19.png}

Após terem encontrado a Terra feita de sal, os tempos passaram. A caminhada continuou, até que chegaram a um lugar diferente… Uns
vastos espaços abertos, totalmente brancos, e os antigos se perguntaram:
``O que é aquilo?''. Pegaram a Terra branca com as mãos e disseram o
nome: ``\emph{yvyku'ix\gr{ῖ}renda}, morada de Terra branca, morada de
\emph{ytaku'i}, areia''.

Após terem encontrado \emph{yvy ku'ix\gr{ῖ}renda}, a morada de Terra branca,
os tempos se passaram. A caminhada continuou, até que chegou a um lugar
onde encontraram \emph{tataryku}, fogo líquido, que havia sido deixado
ali por \emph{Nhamandu Tenondegua} quando criou a Terra. Ao chegar,
\emph{ore retarã} \emph{ypykuery}, nossos antigos parentes, viram que
também lá era impossível permanecer ou formar o \emph{tekoa}, pois era
um lugar muito quente, onde havia montanhas que soltavam
\emph{tataryku}, lavas de vulcão. Observaram de perto o que tinha ali e
decidiram deixar o lugar porque ali mora o espírito do fogo.

Partiram dali e, muito tempo depois, através de sabedoria espiritual,
\emph{ore retarã ypykuery}, nossos antigos parentes, continuaram a
caminhada. Por fim, chegaram à margem do mar e descobriram que lá
sopravam ventos muitos frios e que havia montanhas de gelo e a água era
realmente salgada, e que muitos animais marinhos e aves já habitavam
ali. Então, chamaram o lugar de \emph{mymba retã}, \emph{ypo},
\emph{guyra}, morada de animais e aves marinhas.

Finalmente chegaram ao lugar do sol poente, consagraram o lugar onde
pisaram e encontraram a beira do Mar. A este lugar deram o nome de
\emph{Para yvytu ro'y}, porque \emph{Para} é ``oceano'', \emph{yvytu} é
``vento'', \emph{ro'y} é ``frio''. É \emph{Yro'y rapyta}, a origem do
frio, o Oceano Pacífico.

%% \versal{ILUSTRAÇÃO} 20 (\versal{OCEANO} \versal{PACÍFICO})}

 

Depois de descobrirem este lugar, os tempos passaram. Depois de muito
\emph{mborai}, canto ritual, \emph{xeramõi} e \emph{xejaryi}, nossos
antigos avôs e avós, tiveram uma revelação de \emph{Nhanderu} sobre
\emph{yvy porã}, Terra boa e fértil. \emph{Xeramõi kuéry} reuniram,
então, todos os \emph{jeguakava} \emph{jaxukava porãgue'i} na grande
\emph{Opyre}, casa de rituais.

\imagemmedia{}{./img/20.png}

 

 
