\chapterspecial{Comissão guarani yvyrupa (\versal{CGY})}{}{}
 

%% \versal{ILUSTRAÇÃO} 30 \versal{CIDADE} \versal{ENGOLINDO} A \versal{ALDEIA}}

 

O tempo se passou e eu, que cresci na luta pela Terra para meu povo, fiz
amizade com os demais líderes Guarani, caciques e \emph{xeramõi kuery}.
Juntos, decidimos criar uma organização Guarani que promovesse uma ampla
discussão sobre os lugares para vivermos e mantermos a nossa cultura
milenar, fazendo valer os direitos conquistados na Constituição
Nacional.

A Comissão Guarani \emph{Yvyrupa} nasceu para representar o povo Guarani
no Sul e no Sudeste do Brasil e, também, para fortalecer os contatos com
as lideranças Guarani que vivem nos países que compõem o Mercosul com o
objetivo de lutar pelo reconhecimento das Terras ocupadas pelo nosso
povo e, principalmente, de garantir a demarcação e a regularização
fundiária das nossas Terras ancestrais. Além disso, nós nos unimos para
lutar por saúde, educação e o fortalecimento de nossa cultura.

Temos respaldo constitucional, através do artigo 231, para exigir a
demarcação das Terras indígenas e a garantia do usufruto exclusivo das
Terras tradicionalmente ocupadas pelos povos indígenas. Até hoje, porém,
a grande maioria de nossas Terras não foram reconhecidas nem demarcadas.

Nossa luta é contra grandes obras governamentais e privadas, que têm um
enorme poder de destruição do meio ambiente. Sabemos que a Mata
Atlântica está ameaçada (restam menos de 7 \% de sua cobertura original)
e, com ela, nossas Terras e modo de viver. A natureza é o começo, o meio
e o fim. Sabemos que nossa luta vai ficando cada vez mais difícil, mas
temos esperança de que o Estado Brasileiro possa, de fato, executar o
que está escrito na Carta Magna do país: a demarcação de nossas Terras.
Nutrimos essa esperança pela vida de nosso povo, das nossas crianças e
dos \emph{xeramõi kuery} e \emph{xejaryi kuery}.

 

 
