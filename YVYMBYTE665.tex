\chapterspecial{Yvy mbyte}{Centro da Terra}{}
 

 

%% \versal{ILUSTRAÇÃO} 17} {(\versal{MAR})}

 

\emph{Yvy mbyte é} um lugar sagrado para os \emph{jeguakava} e
\emph{jaxukava} originários. Eles descobriram, através da sabedoria
espiritual, que ali, debaixo da Terra, havia \emph{Yy rupa marãe'y},
lagos de águas eternas, que foram deixadas por \emph{Nhamandu
Tenondegua} quando, da ponta de seu \emph{popygua}, bastão insígnia,
originou a primeira superfície terrestre. Era oca, e dentro dela nasceu
uma grande água subterrânea, nomeada pelos não indígenas de Aquífero
Guarani, berçário dos \emph{jeguakava} e \emph{jaxukava} originários,
atualmente conhecidos pelos \emph{jurua}, não"-indígenas, como Guarani.

\emph{Jeguakava} e \emph{jaxukava} originários se orientavam pelo brilho
dos lagos das águas eternas e, com sua sabedoria espiritual, enxergavam
todas as extremidades de \emph{Yvyrupa}, e descobriram que a Terra era
redonda e que havia um grande mar salgado, \emph{Para guaxu}. E distinguiram cinco direções:


\settowidth{\versewidth}{Yvytu yma, lugar dos ventos originários, frios.}
\begin{verse}[\versewidth]
Yvy mbyte, o centro da Terra;\\
{Ka'arua}, onde o sol se põe;\\
{Tenonde}, onde o sol nasce;\\
{Yvytu katu}, onde se originam os ventos bons;\\
{Yvytu yma}, lugar dos ventos originários, frios.\\
\end{verse}
Essas cinco direções iriam orientar nossos movimentos em \emph{Yvyrupa},
no espaço terrestre.