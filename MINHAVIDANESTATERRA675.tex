%\imagemmedia{}{./img/29.png}

\chapterspecial{Minha vida nesta terra}{}{}
 

%% } {\versal{FUNDO} \versal{VERMELHO}, \versal{LETRAS} \versal{EM} \versal{PRETO}.}

 

%% \versal{ILUSTRAÇÃO} 29 \versal{FOTO} \versal{RETRATO} \versal{DE} \versal{TIMOTEO}}

 

Meu nome em Guarani é \emph{Verá Xunu Popygua}. Nasci em 1969, em
\emph{Yvy mbyte}, hoje conhecida como Tríplice Fronteira. Passei os
primeiros anos da minha infância na aldeia \emph{Tamanduá} na província
de \emph{Misiones}, Argentina. Quando eu tinha sete anos de idade, meu
pai voltou para sua aldeia de origem, no sul do Brasil (Palmeirinha,
\versal{PR}), onde faleceu. Anos mais tarde, enquanto \emph{Guarani Mbya}, tive
curiosidade de conhecer todas as regiões da Mata Atlântica. Quando eu
tinha doze anos, fiz uma viagem a pé, uma caminhada de trinta dias, para
chegar na aldeia Morro da Saudade, atual \emph{Ti Tenonde Porã}, em São
Paulo, Brasil. Nesta aldeia convivi com o \emph{xeramõi} \emph{Karai
Poty}, José Fernandes. Era o ano de 1984 e, neste momento, conheci
também a luta pela Terra.

\emph{Nhanderu} criou a Terra para que possamos todos viver nela. Apesar
de os Guarani viverem na amplidão e sem fronteiras, desde os anos 1970,
muito tempo depois do desaparecimento das bandeiras e dos bandeirantes,
no Estado de São Paulo, onde cresci, os \emph{Guarani Mbya} se viram
novamente obrigados a lutar pela defesa de seu território e pelo
reconhecimento de suas Terras, visando à demarcação das aldeias.

 

Nesse período, as lideranças \emph{Guarani Mbya} do movimento pela Terra
no Estado de São Paulo, onde vivo até hoje, foram os caciques José
Fernandes, Nivaldo Martins da Silva (aldeia Morro da Saudade -- São
Paulo), Altino dos Santos (aldeia Boa Vista -- Ubatuba), Antônio Branco
(aldeia Serra dos Itatins -- Itariri), Jejoko Samuel Bento dos Santos
(aldeia Ribeirão Silveira -- São Sebastião). Entre os jovens aprendizes
desse movimento, estávamos eu e Marcos do Santo Tupã, filho de Altino
dos Santos.

Aos poucos, fui assimilando que a luta pela Terra tinha um grande
significado: garantir futuro para as crianças e afirmar a
autodeterminação de nosso povo. E compreendi que essa luta não era
apenas no Estado de São Paulo, mas em toda extensão de \emph{Yvyrupa}, o
Território Guarani sem fronteiras.

Casei"-me com Florinda Martins da Silva, \emph{Ara Yxapya Yvoty Mirim}.
Tornei"-me cacique da Terra indígena \emph{Tenonde Porã}, em 3 de Março
de 2003, a mesma aldeia em que cheguei em 1984. Na aldeia vizinha,
\emph{Krukutu}, Marcos do Santos Tupã também se tornou cacique. Juntos,
assumimos a continuação da luta pelas Terras \emph{Guarani Mbya}.

 

 
